
%
% Ragnar Ardal
%
% Logic in Computer Science
%
% Reykjavik University 2016
%

\documentclass[11pt,a4paper]{article}

\usepackage{amsmath}
\usepackage{amssymb}
%\usepackage{amsthm}

\usepackage{mathtools}

\usepackage{enumerate}

\newcommand{\chapter}[2]{%
\setcounter{section}{#1}%
\addtocounter{section}{-1}%
\section{#2}%
}
\newcommand{\subchapter}[2]{%
\setcounter{subsection}{#1}%
\addtocounter{subsection}{-1}%
\subsection{#2}%
}
\newcommand{\problem}[1]{%
\setcounter{subsubsection}{#1}%
\addtocounter{subsubsection}{-1}%
\subsubsection{\hfill}%
}
\newcommand{\solution}{%
\subsubsection*{Solution}%
}

\DeclareMathOperator{\T}{\text{\textbf T}}
\DeclareMathOperator{\F}{\text{\textbf F}}
\DeclareMathOperator{\lthen}{\to}
\DeclareMathOperator{\lwhen}{\gets}
\DeclareMathOperator{\lif}{\gets}
\DeclareMathOperator{\liff}{\leftrightarrow}
\DeclareMathOperator{\lxor}{\oplus}

\begin{document}

\title{TA Solutions}
\date{}
\author{}
\maketitle

\chapter{1}{The Foundations: Logic and Proofs}
	\subchapter{1}{Propositional Logic}
		\problem{1}
			Which of these sentences are propositions?
			What are the truth values of those that are propositions?
			\begin{enumerate}[a)]
				\item Boston is the capital of Massachusetts.
				\item Miami is the capital of Florida.
				\item $2 + 3 = 5$.
				\item $5 + 7 = 10$.
				\item $x + 2 = 11$.
				\item Answer this question.
			\end{enumerate}
		\solution{}
			\begin{enumerate}[a)]
				\item This sentence is a proposition and its truth value is $\T$.
				\item This sentence is a proposition and its truth value is $\F$.
				\item This sentence is a proposition and its truth value is $\T$.
				\item This sentence is a proposition and its truth value is $\F$.
				\item This sentence is not a proposition.
				\item This sentence is not a proposition.
			\end{enumerate}
		\problem{3}
			What is the negation of each of these propositions?
			\begin{enumerate}[a)]
				\addtocounter{enumi}{1}
				\item There is no pollution in New Jersey.
				\item $2 + 1 = 3$.
				\item The summer in Maine is hot and sunny.
			\end{enumerate}
		\solution{}
			\begin{enumerate}[a)]
				\addtocounter{enumi}{1}
				\item There is pollution in New Jersey.
				\item $2 + 1 \neq 3$
				\item The summer in Maine is either hot or sunny but not both.
			\end{enumerate}
		\problem{11}
			Let $p$ and $q$ be the propositions
			\begin{itemize}
				\item $p$: It is below freezing.
				\item $q$: It is snowing.
			\end{itemize}
			Write these propositions using $p$ and $q$ and logical connectives (including negations).
			\begin{enumerate}[a)]
				\item It is below freezing and snowing.
				\item It is below freezing but not snowing.
				\item It is not below freezing and it is not snowing.
				\item It is either snowing or below freezing (or both).
				\item If it is below freezing, it is also snowing.
				\item Either it is below freezing or it is snowing,
					but it is not snowing if it is below freezing.
				\item That it is below freezing is necessary and sufficient for it to be snowing.
			\end{enumerate}
		\solution{}
			\begin{enumerate}[a)]
				\item 
				\item 
				\item 
				\item 
				\item 
				\item 
				\item 
			\end{enumerate}
		\problem{13}
			\begin{itemize}
				\item $p$: You drive over 65 miles per hour.
				\item $q$: You get a speeding ticket.
			\end{itemize}
			Write these propositions using $p$ and $q$ and logical connectives (including negations).
			\begin{enumerate}[a)]
				\item You do not drive over 65 miles per hour.
				\item You drive over 65 miles per hour,
					but you do not get a speeding ticket.
				\item You will get a speeding ticket if you drive over 65 miles per hour.
				\item If you do not drive over 65 miles per hour,
					then you will not get a speeding ticket.
				\item Driving over 65 miles per hour is sufficient for getting a speeding ticket.
				\item You get a speeding ticket,
					but you do not drive over 65 miles per hour.
				\item Whenever you get a speeding ticket,
					you are driving over 65 miles per hour.
			\end{enumerate}
		\solution{}
			\begin{enumerate}[a)]
				\item 
				\item 
				\item 
				\item 
				\item 
				\item 
				\item 
			\end{enumerate}
		\problem{17}
			Determine whether each of these conditional statements is true or false.
			\begin{enumerate}[a)]
				\item If $1 + 1 = 2$, then $2 + 2 = 5$.
				\item If $1 + 1 = 3$, then $2 + 2 = 4$.
				\item If $1 + 1 = 3$, then $2 + 2 = 5$.
				\item If monkeys can fly, then $1 + 1 = 3$.
			\end{enumerate}
		\solution{}
			\begin{enumerate}[a)]
				\item 
				\item 
				\item 
				\item 
			\end{enumerate}
		\problem{21}
			For each of these sentences, state what the sentence means if the logical connective or is an inclusive or (that is, a disjunction) versus an exclusive or. Which of these meanings of or do you think is intended?
			\begin{enumerate}[a)]
				\item To take discrete mathematics, you must have taken calculus or a course in computer science.
				\item When you buy a new car from Acme Motor Company, you get a \$2000 back in cash or a 2\% car loan.
				\item Dinner for two includes two items from column A or three items from column B.
				\item School is closed if more than 2 feet of snow falls or if the wind chill is below -100.
			\end{enumerate}
		\solution{}
			\begin{enumerate}[a)]
				\item 
				\item 
				\item 
				\item 
			\end{enumerate}
		\problem{25}
			Write each of these propositions in the form ``$p$ if and only if $q$'' in English.
			\begin{enumerate}[a)]
				\item If it is hot outside you buy an ice cream cone, and if you buy an ice cream cone it is hot outside.
				\item For you to win the conteset it is necessary and sufficient that you have the only winning ticket.
				\item You get promoted only if you have connections, and you have connections only if you get promoted.
				\item If you watch television your mind will decay, and conversely.
				\item The train runs late on exactly those days when I take it.
			\end{enumerate}
		\solution{}
			\begin{enumerate}[a)]
				\item 
				\item 
				\item 
				\item 
				\item 
			\end{enumerate}
		\problem{31}
			Construct a truth table for each of these compound propositions.
			\begin{enumerate}[a)]
				\addtocounter{enumi}{2}
				\item $(p \lor \neg q) \lthen q$
				\addtocounter{enumi}{1}
				\item $(p \lthen \neg p) \liff (p \liff q)$
				\item $(p \liff q) \lxor (p \liff \neg q)$
			\end{enumerate}
		\solution{}
			\begin{enumerate}[a)]
				\addtocounter{enumi}{2}
				\item 
				\addtocounter{enumi}{1}
				\item 
				\item 
			\end{enumerate}
	\subchapter{2}{Applications of Propositional Logic}
		\problem{7}
			Express these system specifications using the propositions $p$ ``The message is scanned for viruses'' and $q$ ``The message was sent from an unknown system'' together with logical connectives (including negation).
			\begin{enumerate}[a)]
				\item ``The message is scanned for viruses whenever the message was sent from an unknown system.''
				\item ``The message was sent from an unknown system but it was not scanned for viruses.''
				\item ``It is necessary to scan the message for viruses whenever it was sent from an unknown system.''
				\item ``When a message is not sent from an unknown system it is not scanned for viruses.''
			\end{enumerate}
		\solution{}
			\begin{enumerate}[a)]
				\item 
				\item 
				\item 
				\item 
			\end{enumerate}
		\problem{11}
			Are these system specifications consistent? ``The router can send packets to the edge system only if it supports the new address space. For the router to support the new address space it is necessary that the latest software release be installed. The router can send packets to the edge system if the latest software release is installed. The router does not support the new address space.''
		\solution{}
\end{document}
