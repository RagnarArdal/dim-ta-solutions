
%
% Ragnar Ardal
%
% Logic in Computer Science
%
% Reykjavik University 2016
%

\documentclass[11pt,a4paper,multicol]{article}

%\usepackage{multicol}

\usepackage{amsmath}
\usepackage{amssymb}
%\usepackage{amsthm}

\usepackage{mathtools}

\usepackage{enumerate}

% Organization of LaTeX file

\newcommand{\chapter}[2]{%
\setcounter{section}{#1}%
\addtocounter{section}{-1}%
\section{#2}%
}
\newcommand{\subchapter}[2]{%
\setcounter{subsection}{#1}%
\addtocounter{subsection}{-1}%
\subsection{#2}%
}
\newcommand{\problem}[1]{%
\setcounter{subsubsection}{#1}%
\addtocounter{subsubsection}{-1}%
\subsubsection{\hfill}%
}
\newenvironment{subproblem}
	{\begin{enumerate}[a)]}
	{\end{enumerate}}
\newcommand{\skipitem}{\addtocounter{enumi}{1}}
\newcommand{\solution}{%
\subsubsection*{Solution}%
}

% Misc math operators

\DeclarePairedDelimiter{\ceil}{\lceil}{\rceil}
\DeclarePairedDelimiter{\floor}{\lfloor}{\rfloor}

% Operators for predicate logic

\DeclareMathOperator{\T}{\text{\textbf T}}
\DeclareMathOperator{\F}{\text{\textbf F}}
\DeclareMathOperator{\lthen}{\to}
\DeclareMathOperator{\limplies}{\to}
\DeclareMathOperator{\lwhen}{\gets}
\DeclareMathOperator{\lif}{\gets}
\DeclareMathOperator{\liff}{\leftrightarrow}
\DeclareMathOperator{\lxor}{\oplus}

% Sets

\DeclarePairedDelimiter{\set}
	{\lbrace}
	{\rbrace}
\DeclareMathOperator{\ZZ}{\mathbb{Z}}
\DeclareMathOperator{\SetOfIntegers}{\ZZ}
\DeclareMathOperator{\ZZPos}{\mathbb{Z}^+}
\DeclareMathOperator{\SetOfPositiveIntegers}{\ZZPos}
\DeclareMathOperator{\NN}{\mathbb{N}}
\DeclareMathOperator{\SetOfNaturalNumbers}{\NN}
\DeclareMathOperator{\RR}{\mathbb{R}}
\DeclareMathOperator{\SetOfRealNumbers}{\RR}
\DeclareMathOperator{\RRPos}{\mathbb{R}^+}
\DeclareMathOperator{\SetOfPositiveRealNumbers}{\RRPos}
\DeclareMathOperator{\QQ}{\mathbb{Q}}
\DeclareMathOperator{\SetOfRationalNumbers}{\QQ}
\DeclareMathOperator{\CC}{\mathbb{C}}
\DeclareMathOperator{\SetOfComplexNumbers}{\CC}

\begin{document}

\title{TA Solutions 4}
\date{}
\author{}
\maketitle

\chapter{2}{Basic Structures: Sets, Functions, Sequences, Sums, and Matrices}
	\subchapter{3}{Functions}

		\problem{3}
			Determine whether $f$ is a function from the set of all bit strings to the set of integers if:
			\begin{subproblem}
				\item $f(s)$ is the position of a 0 bit in $S$.
				\item $f(s)$ is the number of 1 bits in $S$.
				\item $f(s)$ is the smallest integer $i$ such that the $i$th bit of $S$ is 1 and $f(S) = 0$ when $S$ is the empty string, the string with no bits.
			\end{subproblem}
		\solution{}
			\begin{subproblem}
				\item Not a function because (e.g.) a bit string might not have a 0 bit and thus be undefined by $f$.
				\item Is a function because all bit strings (even the empty one) have some amount of 1 bits, even if that amount is 0.
				\item Not a function because any bit string where all the bits are 0 bits, is undefined by $f$.
			\end{subproblem}

		\problem{9}
			Find these values.
			\begin{subproblem}
				\skipitem
				\skipitem
				\item $\ceil{-\frac{3}{4}}$
				\item $\floor{-\frac{7}{8}}$
				\skipitem
				\skipitem
				\item $\floor{\frac{1}{2} + \ceil{\frac{3}{2}}}$
			\end{subproblem}
		\solution{}
			\begin{subproblem}
				\skipitem
				\skipitem
				\item $\ceil{-\frac{3}{4}} = - \floor{\frac{3}{4}} = 0$
				\item $\floor{-\frac{7}{8}} = - \ceil{\frac{7}{8}} = -1$
				\skipitem
				\skipitem
				\item $\floor{\frac{1}{2} + \ceil{\frac{3}{2}}} = \floor{\frac{1}{2} + 2} = \floor{\frac{1}{2}} + 2 = 0 + 2 = 2$
			\end{subproblem}

		\problem{11}
			Which functions in Exercise 10 are onto? (functions from $\set{a, b, c, d}$ to itself)
			\begin{subproblem}
				\item $f(a) = b$,
					$f(b) = a$,
					$f(c) = c$,
					$f(d) = d$
				\item $f(a) = b$,
					$f(b) = b$,
					$f(c) = d$,
					$f(d) = c$
				\item $f(a) = d$,
					$f(b) = b$,
					$f(c) = c$,
					$f(d) = d$
			\end{subproblem}
		\solution
			\begin{subproblem}
				\item Is onto because each element of the image has a preimage.
				\item Not onto because $a$ has no preimage.
				\item Not onto because $a$ has no preimage.
			\end{subproblem}

		\problem{23}
			Determine wheter each of these functions is a bijection from $\RR$ to $\RR$.
			\begin{subproblem}
				\item $f(x) = 2x + 1$
				\item $f(x) = x^2 + 1$
				\item $f(x) = x^3$
				\item $f(x) = (x^2 + 1)/(x^2 + 2)$
			\end{subproblem}
		\solution
			\begin{subproblem}
				\item Yes, images in $\RR$ have a unique preimage, and vice versa. Hence $f$ is a bijection.
				\item No, $f$ is neither one-to-one nor onto and thus not a bijection.
				\item Yes, all real numbers have a unique cube and a unique cube root, and thus $f$ is a bijection.
				\item No, for any $a$ in the domain $f(a) = f(-a)$, and thus $f$ is not a bijection.
			\end{subproblem}

		\problem{63}
			Draw the graph of the function $f(x) = \floor{2x}$ from $\RR$ to $\RR$.
		\solution
			Some fancy plot.

		\problem{69}
			Find the inverse function of $f(x) = x^3 + 1$.
		\solution
			Suppose $y$ is the image of $x^3 + 1$, then
			\begin{align*}
			   			    & y = x^3 + 1 \\
				\Rightarrow & y - 1 = x^3 \\
				\Rightarrow & y - 1 = x^3
			\end{align*}

		\problem{74}
			Prove or disprove each of these statements about the floor and ceiling functions.
			\begin{subproblem}
				\item $\ceil{\floor{x}} = \floor{x}$ for all real numbers $x$.
				\item $\floor{2x} = 2\floor{x}$ whenever $x$ is a real number.
			\end{subproblem}
		\solution
			\begin{subproblem}
				\item
				\item
			\end{subproblem}
			
\chapter{2}{Appendix 2: Exponential and Logarithmic Functions}

		\problem{1}
			Express each of the following quantities as powers of 2.
			\begin{subproblem}
				\item $2 \cdot 2^2$
				\item $(2^2)^3$
				\item $2^{(2^2)}$
			\end{subproblem}
		\solution
			\begin{subproblem}
				\item
				\item
				\item
			\end{subproblem}

		\problem{2}
			Find each of the following quantities.
			\begin{subproblem}
				\item $\log_2 1024$
				\item $\log_2 1/4$
				\item $\log_4 8$
			\end{subproblem}
		\solution
			\begin{subproblem}
				\item
				\item
				\item
			\end{subproblem}
			
		\problem{3}
			Suppose that $\log_4 x = y$ when $x$ is a positive real number.
			Find each of the following quantities.
			\begin{subproblem}
				\item $\log_2 x$
				\item $\log_8 x$
				\item $\log_{16} x$
			\end{subproblem}
			
		\solution
			\begin{subproblem}
				\item
				\item
				\item
			\end{subproblem}

		\problem{4}
			Let $a$, $b$, and $c$ be positive real numbers.
			Show that $a^{\log_b c} = c^{\log_b a}$.
		\solution
			

		\problem{5}
			Draw the graph of $f(x) = b^x$ for all real numbers $x$ if $b$ is
			\begin{subproblem}
				\item 3.
				\item 1/3.
				\item 1.
			\end{subproblem}
		\solution
			\begin{subproblem}
				\item
				\item
				\item
			\end{subproblem}
			
		\problem{6}
			Draw the graph of $f(x) = \log_b x$ for the positive real numbers $x$ if $b$ is
			\begin{subproblem}
				\item 4.
			\end{subproblem}
		\solution
			\begin{subproblem}
				\item
			\end{subproblem}

\chapter{2}{Basic Structures: Sets, Functions, Sequences, Sums, and Matrices}
	\subchapter{4}{Sequences and Summations}

		\problem{1}
			Find these terms of the sequence $\set{a_n}$, where $a_n = 2 \cdot (-3)^n + 5^n$.
			\begin{subproblem}
				\item $a_0$
				\item $a_1$
				\item $a_4$
				\item $a_5$
			\end{subproblem}
		\solution
			\begin{subproblem}
				\item
				\item
				\item
				\item
			\end{subproblem}

		\problem{5}
			List the first 5 terms of each of these sequences.
			\begin{subproblem}
				\item the sequence that begins with 2 and in which each successive term is 3 more than the preceding term
				\skipitem
				\skipitem
				\skipitem
				\item the sequence that lists each positive integer three times, in increasing order
				\item the sequence that lists the odd positive integers in increasing order, listing each odd integer twice
			\end{subproblem}
		\solution
			\begin{subproblem}
				\item
				\skipitem
				\skipitem
				\skipitem
				\item
				\item
			\end{subproblem}

		\problem{9}
			Find the first five terms of the sequence defined by each of these recurrence relations and initial conditions.
			\begin{subproblem}
				\item $a_n = 6a_{n-1}$, $a_0 = 2$
				\skipitem
				\item $a_n = a_{n-a} + 3a_{n-2}$, $a_0 = 1$, $a_1 = 2$
			\end{subproblem}
		\solution
			
\end{document}
