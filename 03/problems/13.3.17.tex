\documentclass[../main.tex]{subfiles}

\begin{document}

\problem{17}
Find the language recognized by the given deterministic finite-state automaton. \\
\includegraphics[width=\textwidth]{img/Q13_3_17}

\solution
The set of bit strings that start with $0$, $10$, or $11$;
i.e., $\{0, 10, 11\}\{0, 1\}^\ast$


\end{document}
