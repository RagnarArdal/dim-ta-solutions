\documentclass[../main.tex]{subfiles}

\begin{document}

\problem{17}
\begin{enumerate}[a)]
	\item Does the line through $P(1, 2, -3)$ with direction vector $\vec{d} = \begin{bmatrix}1&2&-3\end{bmatrix}^T$ lie in the plane $2x - y - z = 3$? Explain.
	\item Does the plane through $P(4, 0, 5)$, $Q(2, 2, 1)$, and $R(1, -1, 2)$ pass through the origin? Explain.
\end{enumerate}

\solution
\begin{enumerate}[a)]
	\item No, even though the point $P(1, 2, -3)$ is on the plane since $2\cdot1 - 1\cdot2 - 1\cdot(-3) = 2 - 2 + 3 = 3$, we see that the dot product of the line's direction vector with the normal of the plane is
		\[
			\begin{bmatrix}1&2&-3\end{bmatrix}^T
			\cdot
			\begin{bmatrix}2&-1&-1\end{bmatrix}^T
			=
			1\cdot2 + 2\cdot(-1) + (-3)\cdot(-1)
			=
			1 - 2 + 3
			=
			2
		\]
		so they are not orthogonal and thus the line cannot lie in the plane.
	\item Start by computing the normal of the plane
		\[
			\overrightarrow{QP}
			=
			\begin{bmatrix}
				4 - 2 \\
				0 - 2 \\
				5 - 1 \\
			\end{bmatrix}
			=
			\begin{bmatrix*}[r]
				2 \\
				-2 \\
				4 \\
			\end{bmatrix*}
			,
			\overrightarrow{RP}
			=
			\begin{bmatrix}
				4 - 1 \\
				0 - (-1) \\
				5 - 2 \\
			\end{bmatrix}
			=
			\begin{bmatrix*}[r]
				3 \\
				1 \\
				3 \\
			\end{bmatrix*}
		\]
		\[
			\vec{n}
			=
			\overrightarrow{QP}
			\times
			\overrightarrow{RP}
			=
			\begin{bmatrix}
				(-2)\cdot3 - 4\cdot1 \\
				-(2\cdot3 - 4\cdot3) \\
				2\cdot1 - (-2)\cdot3 \\
			\end{bmatrix}
			=
			\begin{bmatrix}
				-6 - 4 \\
				-(6 - 12) \\
				2 - (-6) \\
			\end{bmatrix}
			=
			\begin{bmatrix*}[r]
				-10 \\
				6 \\
				8 \\
			\end{bmatrix*}
		\]
		which gives us the equation of the plane
		\[
			\begin{bmatrix*}[r]
				-10 \\
				6 \\
				8 \\
			\end{bmatrix*}
			\cdot
			\begin{bmatrix}
				x \\
				y \\
				z \\
			\end{bmatrix}
			=
			\begin{bmatrix*}[r]
				-10 \\
				6 \\
				8 \\
			\end{bmatrix*}
			\begin{bmatrix}
				2 \\
				2 \\
				1 \\
			\end{bmatrix}
			\Rightarrow
			-10x + 6y + 8z
			=
			0.
		\]
		When we plug the origin into the equation of the plane we get
		\[
			-10\cdot0 + 6\cdot0 + 8\cdot0 = 0 + 0 + 0 = 0
		\]
		so the plane does indeed pass through the origin.
\end{enumerate}

\end{document}
